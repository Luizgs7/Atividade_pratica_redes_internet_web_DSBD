\documentclass{article}
\usepackage[utf8]{inputenc}
\usepackage[T1]{fontenc}
\usepackage[pdftex]{hyperref}

\title{Atividade 01 - Disciplina de Infraestrutura Computacional - Módulo 2}
\author{Luiz Gabriel de Souza}
\begin{document}
\maketitle

\href{https://github.com/Luizgs7/Atividade_pratica_redes_internet_web_DSBD}{Link para o GitHub}

\section{Quais as diferenças na estrutura da rede IPÊ de 2016 (slide 8) para 2020/2021?}
R: É visivel duas mudanças consideráveis:

a) A primeira delas é a velocidade das redes. A que liga São Paulo a Fortaleza, em 2016 era de 10Gbps e passou para 200Gbps em 2020. 
Além disso, a velocidade da conexão entre os Estados do nordeste foi maior do que nas demais regiões. Vários estados dessa região passaram a ter conectividade de 100Gbps entre si.

b) Aumento da densidade da rede, ou seja, maior interconectividade entre as arestas. Boa vista passou a ter uma ligação direta com Fortaleza, bem como Rio Branco com Distrito Federal, para citar apenas alguns exemplos.

\section{Qual a diferença entre Web e Internet?}
R: A Web está contida na Internet. A Web é mais limitade e consiste num conjunto de páginas acessíveis atráves do protocolo HTTP e é acessada por interfaces web, como os navegadores. 
Já a internet é algo mais amplo, que opera e pode ser acessada de diversas formas.

\section{Quais orgãos administram o ponto br (.br) para além do slide 17?}

R: Existem diversos orgãos responsáveis por regular a internet como um todo no brasil. Os domínio ".br" são adminstrados (além dos já citados no slide 17) por orgãos como:

- CETIC.br (Centro Regional de Estudos para o Desenvolvimento da Sociedade da Informação): Tem como objetivo monitorar a adoção das tecnologias de informação e comunicação no Brasil e é um departamento do NIC.
- CEPTRO.br (Centro de Estudos e Pesquisas em Tecnologia de Redes e Operações): Responsável por iniciativas que apoiam ou aperfeiçoam a infraestrutura da Internet no Brasil.
- CEWEB.br (Centro de Estudos sobre Tecnologias Web): Tem como objetivo fomentar a inovação na Web; estimular o melhor uso da Web; mostrar o potencial da Web a diversos segmentos da sociedade e contribuir para a evolução da Web.
- W3C.Brasil (Consórcio World Wide Web): É um consórcio internacional no qual organizações filiadas, uma equipe em tempo integral e o público desenvolvem padrões para a Web.

Além dos que adminstram o ".br", existem outros como o SACI-Adm (Sistema Administrativo de Conflitos de Internet), NTP.br (Protocolo de Tempo para Redes) e INOC-dba (Inter-Network Operation Center Dial By Autonomous System Number).

\section{Quais características do protocolo HTTP descritas na RFC você já conhecia?}

Já conhecia algun erros comuns de acesso como o 403, 404 e 408, bastante comuns em aplicações. 
Conhecia também a existencia do cache, que aumenta a velocidade de resposta de páginas já acessadas anteriormente. 

\section{Qual o motivo de haver 2 chaves diferentes na figura do slide 32?}

O motivo é porque a imagem demonstra o conceita de criptografia end-to-end, onde a primeira chave é uma chave pública, usada pra criptografar os dados que serão enviados. 
Já a segunda, é uma chave privada, que é usada para descriptografar os dados recebidos e que é conhecida apenas pelos 2 envolvidos, remetente e destinatário.

\end{document}
